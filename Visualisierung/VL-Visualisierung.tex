\documentclass[12pt,a4paper,oneside,normalheadings,abstracton,liststotoc,bibtotoc,titlepage,pdftex]{scrartcl}
%\usepackage {ngerman}
\usepackage [ngerman] {babel}
\usepackage[utf8]{inputenc}
\usepackage{float}
\usepackage{graphicx}
\usepackage{longtable}
\usepackage{tabularx}
\usepackage{setspace}
\usepackage{upgreek}
\usepackage{color}
\usepackage{subfig}
\usepackage{gensymb}
\usepackage{amsmath,amsfonts,amssymb}
\usepackage{hyperref}
% math. Symbole und Umgebungen
\onehalfspacing

\begin{document}
\pagenumbering{Roman}

%\extratitle{~}
\titlehead{Universität Leipzig\\Fakultät für Fakultät für Mathematik und Informatik}
\subject{Vorlesungsmitschrift SoSe 11}
\title{Visualisierung}
\author{gehalten von Prof. Dr. Gerik Scheuermann und Dr. Alexander Wiebel\\
Mitschrift von Tobias Mede und Christoph Kämpf}
\date{\today}

\maketitle[1]

\begin{abstract}

\end{abstract}

\tableofcontents
\pagebreak

\pagenumbering{arabic}

\chapter{Visualisierung in Naturwissenschaft und Technik}

\section{Einführung}
Scientific Visualization ist das womit wir uns beschäftigen.
Visualisierungspipeline:


\section{Datenrepräsentation}
\textbf{Prüfungsrelevant:} Aufbau der Datensätze\\
Drei Teile:
-Definitionsmenge = Teilmenge des Beobachtungsraums B$^b$
-Nachbarschaftsrelation
-Funktion der Definitionsmenge

!!Metrik
\subsection{Voronoidiagramm}
Maximal 2n-5 Ecken und 3n-6 Kanten. Delaunay Triangulierung.

\section{Vektoranalysis}


\begin{tabbing}
Modellebenen: \= - komplexe Sachverhalte vereinfachen $\rightarrow$ Simplifikation \\
\> - Illustration \\
\> - mechanisches Modell (der Moleküle) $\curvearrowright$ \= Atome $\rightarrow$ Massenpunkte \\
\> \> Bindungen $\rightarrow$ Federn \\
\> $\overbrace{empirischeModelle}$ \\
\> - mathematische Modellierung \\
\> Ziel: Extrapolation auf andere Vorgänge \\
\end{tabbing}

\subsection{Was sind und wie ermittele ich Strukturdaten?}
Sind meistens experimentelle Untersuchungen die Strukturdaten liefern.
\begin{description}
\item[Strukturdaten:] kartesischer Koordinaten / innere Koordinaten (Bindungslängen, -winkel und Torsionswinkel (ab 4 Atome))\\
2 Atome $\rightarrow$ Bindungslänge \\
3 Atome $\rightarrow$ Bindungslänge + -winkel \\
4 Atome $\rightarrow$ Bindungslänge + -winkel + Torsionswinkel \\
kartesische Koordinaten können aus inneren Koordinaten berechnet werden
\item[Methoden]
\begin{itemize}
\item Röntgenkristallstrukturanalyse (Beugungsmuster + Fouriertransformation)\\
$\hookrightarrow$ aufgrund des Fehlens einer Systematik der Kristallisation $\curvearrowright$ Empirisches Herangehen \\
$\hookrightarrow$ Daten betreffen nur schwere Atome $\curvearrowright$ Fehlen der H-Atome $\curvearrowright$  H-Atome werden dazu gerechnet 
\item Elektronen- / Neutronenbeugung\\
$\hookrightarrow$ Atome werden nicht mit Röntgenstrahlen, sondern mit e$^-$ und Neutronen gebeugt\\
$\hookrightarrow$ Neutronen werden am Kern gebeugt\\
$\hookrightarrow$ nur für relativ kleine Moleküle machbar
\item Mikrowellenspektroskopie\\
$\hookrightarrow$ nur für sehr kleine Moleküle $\rightarrow$ Trägheitsmoment\\
\item NMR-Spektroskopie
\begin{itemize}
\item Karplus-Gleichung ($\rightarrow$ Dupletts, Tripletts, ... ; $\rightarrow$ H-Atome koppeln über max. 3 Bindungen)
\item Nuclear-Overhauser-Effekt (NOE)\\
- indirekte Kopplung entlang von Bindungen $\rightarrow$ liefern Multipletts\\
- direkte Kopplung über den Raum $\rightarrow$ keine Multipletts\\
- Entkopplung $\rightarrow$ Einstrahlung eines Signals mit der zu entkoppelnden Frequenz\\
Intensitätszunahme $\propto$ $\frac{1}{R^6}$ , für koppelndes H-Atom, wenn man die Frequenz des anderen Kopplungspartners einstrahlt\\
- NOE für Abstände von 3-5 \r{A}
\item \textbf{NOE-Distanzen und Karpluswinkel, aber Moleküle meist unter definiert / unbestimmt}
\end{itemize}
\item Sequenzierung der Moleküle, falls möglich (Nukleinsäuren, Proteine)
Berechnung der Energien, falls zu viele Faltungen möglich
\end{itemize}
\end{description}

\subsection{Woher bekomme ich Strukturdaten?}
\begin{description}
\item[\href{http://www.ccdc.cam.ac.uk/products/csd/}{Cambridge Structural Database}]
\begin{itemize}
\item 1965 gegründet
\item enthält nur Strukturdaten niedermolekularer Verbindungen
\item $>$ 280.000 Moleküle
\end{itemize}
\item[\href{http://www.rcsb.org/pdb/home/home.do}{Brookhaven Protein Data Bank (PDB)}]
\begin{itemize}
\item enthält primär Biomakromoleküle (Proteine, Nukleinsäuren, virale Strukturen)
\item Daten aus Röntgenkristallstrukturanalyse, NMR
\item $\sim$ 25.000 Strukturen (teilweise redundante Informationen)
\end{itemize}
\item[Fragmentbibliotheken]
\begin{itemize}
\item zur Erstellung erster Modelle, wenn man keine Struktur hat
\item Durchschnittswerte werden genutzt und daraus ein Molekül zusammen gebaut z.B. ist jede C-C-Einfachbindung 1,54 \r{A} oder der ideale Winkel für sp$^3$ hybrisierte C-Tetraeder 109,5\degree
\item man erstellt zuerst einen Sketch (eine abstrahierte Form des Moleküls)
\item \textbf{WICHTIG:} daraus erhält man nicht die optimale Geometrie, ABER es sind Abschätzungen z.B. des Volumens möglich (Volumen globulärer Proteine aus Dichte $\rho=\frac{m}{V}$ berechenbar, mit $n=\frac{m}{M}$ und $V=\frac{4}{3}\Pi r^3$)
\end{itemize}
\end{description}

\subsection{Was gibt es für Modelltypen?}
\begin{itemize}
\item Kugel-Stab Modell
\item Kalottenmodell
\item Dreiding Modell
\end{itemize}
Ziel der Modellierung ist es Eigenschaften der Moleküle berechnen zu können z.B. Bildungswärmen, IR-Spektren), hierfür ist die Theorie teils recht komplex. Der Computer hilft zuerst bei der Berechnung dieser Eigenschaften und dann bei der visuellen Darstellung.

\section{Theoretische Grundlagen der Molekülmodellierung/ Quantenmechanik}

Die Grundlage der Molekülmodellierung ist die Quantenmechanik, deren Grundlagen in den Jahren 1925-1935 gelegt wurden. Die Anwendung der Quantenmechanik auf die Chemie nennt man Quantenchemie. Bei der Beschreibung von Molekülen mit Hilfe der Quantenmechanik spricht man auch von der Molekülphysik.

\subsection{Quantenchemie}

Atome im Sinne der Quantenchemie sind Ansammlungen von \textbf{n} Elektronen um einen Kern. Moleküle sind Ansammlungen von \textbf{n} Elektronen und \textbf{N} Kernen. Die Quantenchemie beruht wie alle Theorien auf einem Satz basaler Postulate:
\subsubsection*{1. Postulat}
Alle Bewegungen quantenmechanischer Objekte müssen berechenbar sein $\curvearrowright$ es existiert eine Zustand- / Wellenfunktion die diese Bewegungen beschreibt.
\begin{align}
\Psi &=(x,y,z,t)& \text{x,y,z} &= \text{kartes. Koordinaten} \\
\notag && \text{t} &= \text{Zeit}
\end{align}
Die Wahrscheinlichkeit ein bestimmtes Objekt an einer bestimmten Stelle im Raum anzutreffen entspricht dem Quadrat der Wellenfunktion.
\begin{align}
\Psi^2 &=(x,y,z)& \text{t entfällt zur Vereinfachung}
\end{align}

\subsubsection*{2. Postulat}
Physikalische Größen werden in der Quantenmechanik durch einen Operator repräsentiert. Dieser Operator ist eine Rechenvorschrift die auf $\Psi$ anzuwenden ist um den richtigen Wert der physikalischen Größe zu erhalten. Solche Operatoren heißen \textit{Observable}. Da die Quantenmechanik für große Massen die Gesetze der klassischen Physik wiedergeben muss, konnte dies als Voraussetzung genutzt werden um die Operatoren abzuleiten.\\
In der Quantenmechanik wird jeder \textit{Observablen} (d.h. jeder meßbaren Größe) ein Operator
zugeordnet. Man geht dabei von den Observablen Ort $\widehat{r}$ (mit den Komponenten x, y, z)
und Impuls $\widehat{p}$ (mit den Komponenten $p_x$, $p_y$, $p_z$) aus.

\begin{align}
\widehat{p_x} &= \dfrac{h}{2\Pi i} \dfrac{\delta}{\delta x}\\\\
\notag & \text{muss hier nicht -i über dem Bruchstrich stehen?}\\
\widehat{E} &= \widehat{H} = \widehat{T} + \widehat{V} = - \dfrac{\hbar^2}{2m}(\dfrac{\partial^2}{\partial x^2}\dfrac{\partial^2}{\partial y^2}\dfrac{\partial^2}{\partial z^2})+\widehat{V}(x,y,z)
\end{align}
\begin{description}
\item[$\widehat{p_x}$]-Impulsoperator der x-Koordinate
\item[$\widehat{x}$]-Ortsoperator (dies sind die kartes. Koordinaten)
\item[$\widehat{E}$]-Energieoperator
\item[$\widehat{H}$]-Hamiltonoperator
\item[$\widehat{T}$]-Operator der kinetischen Energie
\item[$\widehat{V}$]-Operator der potentiellen Energie
\item[$i$]-imaginär
\item[$\hbar$]- $\dfrac{h}{2\Pi}$
\end{description}

\subsubsection*{3. Postulat}
Alle möglichen Werte die eine Observable annehmen könnte, ergeben sich aus den Lösungen der Eigenwertgleichung:
\begin{align}
\widehat{A}\Psi _n = A_n \Psi _n
\end{align}
\begin{description}
\item[$\Psi _n$]-Zustandsfunktion/Eigenfunktion
\item[$\widehat{A}$]-Operator der Observablen
\item[$A_n$]-Wert der Observablen/Eigenwerte; Werte für $A_n$ sind nicht kontinuierlich, sondern diskret aufgrund der zugrunde liegenden Quantentheorie.
\item[$\Psi _n$]-es gibt unendlich viele Zustandsfunktion, ABER die Funktion $\Psi ^2$ beschreibt die Wahrscheinlichkeit des Auftretens bestimmter $\Psi _n$
\end{description}

\subsubsection{Schrödingergleichung und ihre Formen}
Die Anwendung des Hamiltonoperator auf die Zustandsfunktion ergibt folgende Gleichung:
\begin{align}
\widehat{H}\Psi _n &= E_n \Psi _n & \rightarrow \text{Grundform der Schrödinger-Gleichung}
\end{align}
$\widehat{H}$ wird für ein System mit N Kernen und n Elektronen zu:
\begin{align}
\widehat{T} &= - \dfrac{\hbar^2}{2m_e} \overset{n}{\underset{i=1}{\sum}} {\nabla_i}^2 - (\dfrac{\hbar^2}{2}) \overset{N}{\underset{I=1}{\sum}} \dfrac{{\nabla_I}^2}{M_I} &\\
\notag && \text{i} &= \text{Zähler für Elektronen} \\
\notag && \text{I} &= \text{Zähler für Kerne}\\
\notag && \nabla &= \text{Nablaoperator}\\
\notag && \Delta &= \nabla^2 = \text{Laplaceoperator}
\end{align}

\begin{align}
\widehat{V} =& -e^2 \overset{n}{\underset{i=1}{\sum}} \overset{N}{\underset{I=1}{\sum}} \dfrac{Z_I}{r_{iI}} & \rightarrow \text{Kern-Elektronen-Anziehung} \\
\notag & +e^2 \overset{n-1}{\underset{i=1}{\sum}} \overset{n}{\underset{j=i+1}{\sum}} \dfrac{1}{r_{ij}} & \rightarrow \text{Elektronen-Elektronen-Abstoßung} \\
\notag & +e^2 \overset{N-1}{\underset{I=1}{\sum}} \overset{N}{\underset{J=I+1}{\sum}} \dfrac{Z_I*Z_J}{R_{IJ}} & \rightarrow \text{Kern-Kern-Abstoßung}
\end{align}
Da $\widehat{H} = \widehat{T} + \widehat{V}$ gilt, kann man nun $\widehat{H}$ schreiben als:
\begin{align}
\widehat{H} =& \dfrac{\hbar^2}{2m_e} \overset{n}{\underset{i=1}{\sum}} {\nabla_i}^2 - (\dfrac{\hbar^2}{2}) \overset{N}{\underset{I=1}{\sum}} \dfrac{{\nabla_I}^2}{M_I}-e^2 \overset{n}{\underset{i=1}{\sum}} \overset{N}{\underset{I=1}{\sum}} \dfrac{Z_I}{r_{iI}}+e^2 \overset{n-1}{\underset{i=1}{\sum}} \overset{n}{\underset{j=i+1}{\sum}} \dfrac{1}{r_{ij}}+e^2 \overset{N-1}{\underset{I=1}{\sum}} \overset{N}{\underset{J=I+1}{\sum}} \dfrac{Z_I*Z_J}{R_{IJ}}
\end{align}
Für H-Atom ist die Schrödingergleichung exakt lösbar. Würde die Elektronen-Elektronen-Abstoßung entfallen, dann würde $H=h_1+h_2+\dots$ die Berechnung des Hamiltonoperators in die Berechnungen von Einzelteilchenoperatoren ($h_1,h_2,\dots$) zerlegbar werden.\\
Lösungen der Schrödingergleichung $\widehat{A_n} \Psi_n = A_n \Psi_n $ für H-Atome sehen generell so aus:\\
\begin{align}
\Psi_n& = R(r) \cdot \Phi(\vartheta, \phi)\\
\notag & \overbrace{\text{Radialteil}} \overbrace{\text{Winkelanteil}}
\end{align}
Diese Lösungen liefern $\infty$ $\Psi_n$ und $\infty$ diskrete $A_n$. $\Psi_n$ besteht aus Radial- und Winkelteil, diese variieren nur an drei verschiedenen Stellen (n, l, m), diese Stellen werden Quantenzahlen (Haupt-, Neben-, Magnetquantenzahl) genannt. Die Spinquantenzahl wurde postuliert, da man beobachtete, dass die Spektrallinien (beim H-Atom) in zwei Linien aufgespalten sind. Dies kommt daher, dass jeder Zustand (n, l, m) energetisch entartet ist.
\begin{align}
\Psi_n &= A_n \cos^m \phi \sin^l \vartheta 
\end{align}
Die Bereiche der Quantenzahlen ergeben sich zu:
\begin{align}
& n \in \{1, 2, 3, \dots \} && \rightarrow \text{Hauptquantenzahl}\\
& l \in \{0, 1, 2, 3, \dots, n-1\} && \rightarrow \text{Nebenquantenzahl}\\
& m \in \{-l, \dots, 0, \dots, l\} && \rightarrow \text{Magnetquantenzahl}\\
& s \in \{\frac{1}{2}, -\frac{1}{2}\} && \rightarrow \text{Spinquantenzahl}
\end{align}
Einelektronenwellefunktionen heißen auch \textbf{Orbitale}, diese $\Psi_n$ beschreiben die Zustände von Elektronen. Die Schrödingergleichung ist schon für das He-Atom nicht mehr analytisch lösbar, weil die Elektronen-Elektronen-Abstoßung nicht aufteilbar ist. (Die Frage welches Elektron wieviel zur Abstoßung beiträgt ist nicht beantwortbar.)

\subsubsection{Näherungen der Schrödingergleichung}
\paragraph{Näherung nach Hartree}
Hartree schlug vor die Elektronen-Elektronen-Abstoßung zu vernachlässigen. Dadurch konnte die Wellenfunktion als Produkt der Einzelwellenfunktionen formuliert werden.
\begin{align}
\Psi(1,2,3,\dots,n) = \phi(1) \cdot \phi(2) \cdot \phi(3) \dots \phi(n) \text{mit:} \widehat{H}=\widehat{h_1}+\widehat{h_2}+\dots
\end{align}
Jeder $h_x$ wirkt nur auf die Wellenfunktionen auf die er ``Anspruch'' hat, da er nur die Koordinaten eines Teilchens kennt sind die anderen Teilchen für ihn Konstanten. Die Möglichkeit den $\widehat{H}$ in eine Summe von Einelektronenoperatoren zu schreiben, ergibt sich erst wenn man die Elektronenwechselwirkung komplett vernachlässigt.\\
Hartree wählte zuerst einen Produktansatz, aber Fock wies daraufhin, dass damit die ``Ununterscheidbarkeit der Teilchen'' verletzt wurde. Diese besagt, dass sich $\Psi$ nicht ändern darf, wenn man zwei ${e^\circleddash}$ vertauscht.
\begin{align}
e(1,2,\dots,i,j,\dots,n)&=\Psi(1,2,\dots,i,j,\dots,n)^2\\
\notag &=\Psi(1,2,\dots,j,i,\dots,n)^2\\
\notag \text{dies trifft dann zu, wenn gilt:}\\
\Psi(1,2,\dots,i,j,\dots,n)&=\Psi(1,2,\dots,j,i,\dots,n) & \rightarrow symmetrisch\\
\Psi(1,2,\dots,i,j,\dots,n)&=-\Psi(1,2,\dots,j,i,\dots,n) & \rightarrow antisymmetrisch
\end{align}
Teilchen mit halbzahligem Spin (${e^\circleddash}$) müssen immer dem Antisymmetrieprinzip genügen. Dem Hartreeprodukt folgend genügen zwei Elektronen (1,2) nicht dem Antisymmetrieprinzip:
\begin{align}
\Psi(1,2)=\alpha(1)\phi_1(1)\beta(2)\phi_1(2)\neq-\Psi(1,2)
\end{align}
ABER eine Kombination aus Hartree-Produkten genügt dem Antisymetrieprinzip:
\begin{align}
\Psi(1,2)&=\alpha(1)\phi_1(1)\beta(2)\phi_1(2)-\alpha(2)\phi_1(2)\beta(1)\phi_1(1)\\
\notag &=\alpha(2)\phi_1(2)\beta(1)\phi_1(1)-\alpha(1)\phi_1(1)\beta(2)\phi_1(2)\\
\notag &=-\Psi(1,2)
\end{align}
Das Ergebnis der obigen Gleichung ändert einfach nur sein Vorzeichen, da das $\Delta$ der beiden Produkte gleich bleibt. Man kann die einzelnen Faktoren aus denen $\Psi$ zusammengesetzt ist in Matrizen überführen und daraus Determinanten berechnen (Slater-Dterminanten). Diese Determinanten haben dann ein ähnliches Aussehen wie die obigen Gleichungen.
\begin{align}
\int det \mid \Psi^*\mid \widehat{H} det \mid \Psi\mid = E
\end{align}

\paragraph{Zentralfeldtheorie/ Modell der unabhängigen Teilchen}
Man berechnet die Durchschnittspotentiale aller ${e^\circleddash}_{2-m}$ mit denen ${e^\circleddash}_1$ wechselwirkt und berechnet dann die Wechselwirkung zwischen ${e^\circleddash}_1$ und dem Durchschnitt von ${e^\circleddash}_{2-m}$. $\curvearrowright$ \\
\begin{description}
\item[Problem:] Berechnung des Durchschnittspotentials, da man dafür $\Psi_n$ für alle ${e^\circleddash}_{2-m}$ bräuchte.
\item[Vorteil:] man bleibt im Modell der unabhängigen Teilchen und kann mit Einteilchenwellenfunktionen arbeiten.
\item[Lösung:] man nimmt zuerst die Einteilchenwellenfunktionen nach Hartree an und berechnet damit die Durchschnittspotentiale und die $\Psi_n$ für jedes ${e^\circleddash}$. Daraus ergeben sich neue $\Psi_n$ mit denen man dann die Durchschnittspotentiale wieder neu berechnet. Das ganze wird solange wiederholt, bis sich die Durchschnittspotentiale bei Neuberechnung nur noch unwesentlich ändern. Dieses Verfahren heißt \textbf{``self consistent field''} (SCF) und liefert für den häufigen Fall, das die $\Psi_n$ \textbf{konvergieren} brauchbare Ergebnisse.
\end{description}
\paragraph{Born-Oppenheimer-Approximation}
Die Born-Oppenheimer-Approximation wird bei Molekülen angewendet und besagt:
\begin{itemize}
\item die meisten Eigenschaften von Molekülen hängen nur von den ${e^\circleddash}$ ab.
\item die Bewegungen der Kerne sind viel langsamer, als die der ${e^\circleddash}$.
\end{itemize}
Für Moleküleigenschaften die nur von den ${e^\circleddash}$ abhängen, kann man die Kerne als ruhend annehmen. Aber wenn sich die Kerne bewegen, kann man annehmen, dass sich die Elektronen ``sofort anpassen''. Damit kann man die Berechnung der Kernbewegung von der Elektronenbewegung abkoppeln.
\begin{align}
\widehat{H} &= \widehat{H}_{el} + \widehat{H}_K \curvearrowright\\
E_T&=E_{el}+E_K \\
\notag E_T &- \text{Totalenegrie}\\
\notag E_{el} &- \text{Energie der Elektronen (berechnet mittels Schrödingergleichung)}\\
\notag E_K &- \text{Energie der Kerne für fixe Kerne (leicht) berechnet}
\end{align}

\subsubsection{Anwendung auf Molekülberechnungen}
Die Kernpositionen können experimentell, aus Fragmentbibliotheken oder über die Berechnung der Schrödingergleichung für verschiedene Kernpositionen bestimmt werden.\\
Geometrieoptimierung = Energieminimierung\\
Die Energie ist eine Funktion der Kernkoordinaten. Die Energie eines Moleküls ist also von der Position aller N Kerne abhängig. Die n-dimensionale Funktion die diese Abhängigkeit widerspiegelt wird als \textbf{Potentialhyperfläche} bezeichnet. Auf dieser Potentialhyperfläche repräsentieren alle Minima (lokale und globale) Isomere des Moleküls.\\
Die Lösungen der Schrödingergleichung für die Elektronen im Molekül (alle ${e^\circleddash}$ des Moleküls beeinflussen sich gegenseitig) ergeben die Molekülorbitale. $\curvearrowright$ MO-Modell. Die Molekülorbitale können als Linearkombination der Atomorbitale mit komplexen Faktoren beschrieben werden \textbf{LCAO-MO} (linear combination of atomic orbitals to molecular orbitals)
\begin{align}
\Psi = c_1 \phi_1 + c_2 \phi_2 + \dots
\end{align}
Aus dieser Gleichung ergeben sich die Wellenfunktionen ($\Psi_n$) die für die Berechnung der Molekülorbitale genutzt werden. Es werden dann die $c_x$ gesucht für die die Energie minimal wird.\\
Für Moleküle arbeitet man dann mit \textbf{HF-LCAO-MO} (Hartree-Fock - linear combination of atom orbitals - molecular orbitals). Für große Moleküle wird die Slater-Determinante sehr groß und es sind zu viele Integrale zu berechnen. Alle Verfahren die mit HF-LCAO-MO funktionieren heißen \textbf{ab initio}.\\
\textbf{Die Born-Oppenheimer-Approximation erhält erst den Begriff Struktur für die Quantenmechanik. Es gibt zwei Optimierungen in der Quantenmechanik:
\begin{enumerate}
\item SCF-Optimierung (${e^\circleddash}$)
\item Geometrieoptimeirung (Kerne) $\rightarrow$ Potentialhyperfläche
\end{enumerate}}
Für jeden möglichen Zustand $\Psi_n$ \textcolor{red}{(dabei sind nicht die unendlich vielen gemeint, sondern man wird wir wohl eher bis zum LUMO gehen??)} eines Atoms muss mindestens ein $\phi$ in der LCAO enthalten sein.\\
\begin{description}
\item[Minimalbasissatz] enthält genau dieses eine $\phi$ pro Zustand $\Psi_n$ des Atoms
\item[Double-Zeta-Basissätze] enthalten zwei $\phi$ pro Zustand $\Psi_n$ des Atoms
\item[Triple-Zeta-Basissätze] enthalten drei $\phi$ pro Zustand $\Psi_n$ des Atoms
\end{description}
Um Rechenzeit zu sparen nutzt man Double- und Triple-Zeta-Basissätze nur für die Valenzschalen der Atome eines Moleküls, da diese für die Reaktivität verantwortlich sind. Die darunter liegenden Schalen werden durch $\Psi_n$ mit Minimalbasissatz beschrieben.\\
Slater hat herausgefunden, dass man um weitere Rechenzeit zu sparen die LCAO durch leichter integrierbare Funktionen durch sogenannte \textbf{Slater-Type-Orbitale} (STO) ersetzen kann. Diese haben folgende Form:
\begin{align}
A e^{-\phi r}y_{l,m}
\end{align}
Eine weitere Verbesserung der Berechenbarkeit erbrachte die Ersetzung der STO durch Linearkombinationen von Gauß-Funktionen, diese sehen wie folgt aus:
\begin{align}
A e^{\phi r^2}y_{l,m}
\end{align}
Die Integration einer Linearkombination aus 7 Gauß-Funktionen ist schneller als die Berechnung einer STO.\\
\textbf{Mit diesem Modell ist die Hartree-Fock-Energie berechenbar, aber diese unterscheidet sich selbst bei genauester Berechnung von der realen Energie. Dieser Unterschied heißt Korrelationsenergie.}
\subsubsection{Korrelationsenergie}
Elektronenkorrelation besteht aus zwei Komponenten:
\begin{description}
\item [Elektronenaustauschkorrelation] nur ${e^\circleddash}$ mit entgegengesetztem Spin dürfen in einem MO sein
\item [Coulomb-Korrelation] ${e^\circleddash}$ mit gleicher Orbitalfunktion dürfen nicht am gleichen Ort sein
\end{description}
Die Korrelationsenergie muss immer dann beachtet werden, wenn die elektronischen Umgebungen zwischen zwei Zuständen stark varriieren z.B. bei der Beschreibung von Reaktionen und bei Strukturproblemen. In solchen Fällen ist der HF-Anstaz nicht ausreichend, dann muss die Korrelationsenergie berechnet werden. Diese Berechnung kann z.B. mittels M\o ller-Plesset erfolgen.

\section{Semi-empirische Verfahren}
Diese Verfahren lösen auch die Schrödingergleichung, berücksichtigen dabei aber nur die Valenzelektronen und arbeiten mit der Born-Oppenheimer-Approximation und allem was da so dranhängt (z.B. HF-LCAO). Diese Methoden sind heute nur noch Überblicks-methoden. Zero-Differential-Overlap ist eine Methode die der Reduktion der zu berechnenden Integrale (meistens Elektronen-Elektronen-Abstoßung betreffend) bei der Lösung der Schrödingergleichung dient.\\
Heute genutzte Methoden:
\begin{itemize}
\item Dewar
\item AM1 (Austin model 1)
\item MNDO (Modified Neglect of Differential Overlap)
\item PM3 (Parameterized Model number 3)
\end{itemize}

\section{Molekülmechanik}
Die Molekülmechanik wird für große Moleküle genutzt die mittels ``ab initio''-Verfahren nicht berechenbar sind. Dabei kehrt man wieder zur klassischen Physik zurück um diese Moleküle zu beschreiben. Dabei werden Atome als \textbf{Massenpunkte} und Bindungen als \textbf{Federn} dargestellt. Es wird eine optimale Sturktur definiert (ideale Bind-, Torsionswinkel und Bindungslängen) und bei Abweichungen davon wird der energetische Unterschied berechnet. Mittels der Molekülmechanik können keinerlei Aussagen über die elektronischen Eigenschaften eines Moleküls getroffen werden. Man kann nur strukturelle Aussagen machen.\\
Für molekülmechanische Berechnungen werden die Energien der bindenden Parameter (Bindungslängen $\mathbf{E_{AB}}$, Bindungswinkel $\mathbf{E_{ABC}}$ und Torsionswinkel $\mathbf{E_{ABCD}}$) als auch die Energien der nichtbindenden Parameter (elektrostatische Wechselwirkungen zwischen partiell geladenen Atomen $\mathbf{E_{el}}$ und London-van der Waals-Kräfte (Elektronenkorrelation) $\mathbf{E_{vdw}}$) benötigt. Dabei können die bindenden Parameter solange verändert werden, solange der Energieverlust durch den Energiegewinn der nichtbindenden Parameter wieder wettgemacht wird.\\
Um auf diese Daten und Annahmen ein sinnvolle Stukturvorhersage aufbauen zu können, müssen geeignete Energiekostenfunktionen mit Hilfe der klassischen Physik gesucht werden. Diese Funktionen sehen allgemein so aus:
\begin{align}
\Delta E &= E_{AB} + E_{ABC} + E_{ABCD} + E_{el} + E_{vdw}\\
\Delta E_{AB} &= k(\Delta R)^2\\
\notag &= k(R(AB)-R_0(AB))^2\\
\notag & \rightarrow \text{Hooksches Gesetz}\\
\notag & \rightarrow \text{k aus Bindungsenergie ermittelbar und aus IR-Spektroskopie (Formel)}\\
\Delta E_{ABC} &= k_{ABC}(\alpha(ABC)-\alpha_0(ABC))\\
\notag & \rightarrow E_{ABC} \text{ aus Deformationsschwingung aus IR-Spektroskopie ableitbar}\\
\Delta E_{ABCD} &= v_T = \overset{J}{\underset{j=1}{\sum}} \dfrac{1}{2} k_j (1+\cos (j\omega-\gamma))\\
\notag E_{ABCD} &-\text{ kann durch Kombination von cos-Funktionen angenähert werden}\\
\notag \omega &-\text{ Drehwinkel}\\
\notag \gamma &-\text{ Nullpunktverschiebung}\\
\notag & \text{das ist die allgemeine Form, meistens reichen jedoch drei Glieder } \curvearrowright\\
v_T &=\dfrac{k_1}{2}(1+\cos \omega)+\dfrac{k_2}{2}(1+\cos 2 \omega)+\dfrac{k_3}{2}(1+\cos 3 \omega)
\notag \dfrac{k_x}{2}\\
\notag &-\text{ Potentialterme (Amplituden der Schwingung)}\\
\notag (1+\cos \mathbf{X}\omega) &-\text{ beeinflussen die Anzahl der Schwingungen pro 360 Grad}
\end{align}
Die Werte für $k_x$ sind nur selten empirisch bestimmbar, deshalb werden sie ab initio berechnet.
\begin{align}
\Delta E_{el} &= \dfrac{1}{4 \pi \epsilon_0}\underset{AB}{\sum} \dfrac{q_A \cdot q_B}{R_{AB}} && \rightarrow \text{Coulombsches Gesetz}
\end{align}
\textcolor{red}{Beziehen sich die Werte für $q_x$ auf die Gesamtladungen der Atome??} $q_x$ wird quantenmechanisch berechnet und gemittelt.
\begin{align}
\Delta E_{vdw} &= -\dfrac{A_{AB}}{R_{AB}^6} \dfrac{B_{AB}}{R_{AB}^{12}}  \rightarrow \text{Lenard-Jones-Potential}\\
\notag A_{AB} &- \text{Konstante aus Atompolarisierbarkeit berechenbar, empirisch ermittelt}\\
\notag B_{AB} &- \text{Empirisch ermittelte Konstante}\\
\notag -\dfrac{A_{AB}}{R_{AB}^6} &- \text{Attraktive Energie ändert sich mit $r^{-6}$}\\
\notag \dfrac{B_{AB}}{R_{AB}^{12}} &- \text{Repulsive Energie ändert sich mit $r^{-12}$}
\end{align}
Die attraktive Energie beginnt früher einen Einfluss auf zwei Atome auszuüben, wird aber bei sehr kleinen R-Werten von der repulsiven Energie, die dann schneller wächst, überholt.\\
Die obenstehenden $\Delta E_x$ sind die energetischen Komponenten die ein Kraftfeld ausmachen. Mit Hilfe dieser Energien und der geschickten Wahl der darin enthaltenen Parameter kann man ein Kraftfeld modellieren.

\subsection{Methoden der Kraftfeldmodellierung}
Alle Konstanten die für die Modellierung eines Kraftfeldes benötigt werden, können auf die experimentell bestimmten Werte gefittet werden. Der Ursprung der Daten auf die gefittet wird bestimmt für welche Molekültypen das Kraftfeld gute Vorhersagen machen kann. Aufgrund der Gleichung des Kraftfeldes muss die Geometrie des Moleküls für alle bindenden und nicht-bindenden Parameter optimiert werden. Der Vorteil hierbei ist die einfache Berechenbarkeit der 1. Ableitung der Kraftfeldgleichung. Die erhaltene Energie lässt \textbf{keine Aussagen} über das Ergebnis zu, weil dieses Ergebnis mittels der klassischen Mechanik berechnet wurde und nicht mittels Quantenmechanik sind die berechneten Energien ``fiktionale'' Werte. ABER ein Vergleich verschiedener Energiewerte die mit der gleichen Krfatfeldgleichung berechnet wurden, erlaubt Aussagen über die Stabilitäten der unterschiedlichen Konformationen. Außerdem ist wichtig zu wissen, dass es \text{kein Kraftfeld gibt, dass universell anwendbar ist}. In den Kraftfeldern werden oft unterschiedliche Parameter für unterschiedliche Zustände eines Atoms (sp$^3$-C, sp$^2$-C, sp-C, benzolisches-C) verwendet, deshab werden für jede Anwendung angepasste Kraftfelder genutzt.\\
Eine Übersicht über verschiedene Kraftfelder:
\begin{description}
\item[MM2-4] nicht für Peptide, aber organische Moleküle geeignet
\item[MMFF] Merck-Kraftfeld, nur für organische Moleküle
\item[OPLS] nur für organische Moleküle
\item[AMBER, CHARMM, GROMOS] gut für Peptide, nicht für organische Moleküle 
\end{description}
``United atom'' Modelle vereinen mehrere Atome rechnerisch zu einem Atom und parametrisieren diese Pseudoatome (z.B. CH-, CH$_2$-, CH$_3$-Gruppen). ``Single atom'' Modelle beziehen in ihre Berechnungen jedes Atom einzel ein, aber berücksichtigen die Umgebung des Atoms für dessen Parametrisierung (sp-C, sp$^2$-C, sp$^3$-C).

\subsection{Methoden zur Berechnung von Moleküleigenschaften}
\begin{description}
\item[ab initio] Lösung der Schrödingergleichung $\rightarrow$ HF-LCAO-MO + Korrelationsenergie
\begin{itemize}
\item VB (valence bond Methode) Alternative zur MO-Theorie, man nutzt vorgefertigte Bindungsorbitalwellenfunktion zur Lösung der Schrödingergleichung
\item DFT (density functional theory) hierbei wird die angenommen, dass die Grundzustands\textbf{energie} aus der Grundzustands\textbf{elektronendichte} bestimmt werden kann (ohne Kenntnis der Grundzustands\textbf{funktion})
\end{itemize}
\item[semi-empirisch] Merck-Kraftfeld, nur für organische Moleküle
\item[Molekülmechanik] gut für Peptide, nicht für organische Moleküle 
\end{description} 
Heutzutage werden Methoden genutzt die DFT und HF-LCAO kombinieren.

\subsection{Methoden zur Geometrieoptimierung}
Um von einer Ausgangsgeometrie zur (im Rahmen des gewählten Modells) optimalen Geometrie zu kommen gib es verschiedene Möglichkeiten:
\begin{itemize}
\item Schrittweise Anpassung der Koordinaten entsprechend der Energiewerte der Teilenergien (bildlich ein langsames spiralförmiges Absteigen in den Faltungstrichter)
\item Gradientenmethoden, man ändert seine Koordinaten entsprechend der Maximierung von $\dfrac{dE}{dx}$ (schnelles direktes Absteigen in den Faltungstrichter)
\item Newton-Verfahren/ Krümmungsverfahren $\dfrac{dE}{dx}$,  $\dfrac{d^2E}{dx^2}$ (zweite Ableitungen schwer oder nicht berechenbar, deshalb oft Anwendung von Quasi-Newton-Verfahren die keine zweite Ableitung benötigen)
\end{itemize}

\subsubsection{Potentialhyperflächen}
Potentialhyperflächen sind n+1-dimensional. Darauf sucht man nach Energieminima, den Minimumsübergangsenergien zwischen diesen Minima und den sich daraus ergebenden Geschwindigkeitskonstanten. Übergangszustände entsprechen Sattelpunkten auf der Potentialhyperfläche. Die Berechnungen des Moleküls entlang des Minimumenergiewegs, zwischen zwei Minima über einen Sattelpunkt, beschreibt die Veränderungen während einer chemischen Reaktion. Das Auffinden der Übergangszustände ist mathematisch sehr schwierig.\\
Experimentell kann man den Übergangszustand (ist durch Schwingungen gekennzeichnet) durch die IR-Spektroskopie bestimmen ($\dfrac{d^2E}{dx^2}$ wird durch die Schwingungsfrequenz repräsentiert). Ein Molekül hat 3N-6 Freiheitgrade, deshalb gilt:
\begin{enumerate}
\item alle 3N-6 $\dfrac{d^2E}{dx^2}$ positiv sind $\rightarrow$ Minimum
\item ein $\dfrac{d^2E}{dx^2}$ negativ ist $\rightarrow$ Sattelpunkt
\item mehr als ein $\dfrac{d^2E}{dx^2}$ negativ sind $\rightarrow$ was anderes
\end{enumerate}
Viele chemische Eigenschaften werden als statistisches Mittel aller Einzelmoleküle angegeben, dies berücksichtigt die Dynamik eines Moleküls.

\subsection{Moleküldynamik}
Nullpunktschwingungsenergie besagt, dass die Kerne in Molekülen selbst im Grundzustand einen gewissen Schwingungsenergiebeitrag haben.\\
Moleküle, Atome in Molekülen beginnen zu schwingen\\
$\rightarrow$ Rotationsenergie $\rightarrow$ Trägheitsmomente \\
$\rightarrow$ Energieübertragung zwischen Molekülen (Zusammenstöße)\\
$\curvearrowright$ Berechnung der Trajektorien
\paragraph{Grundlage}
\begin{align}
F &= m \cdot a = m \cdot \dfrac{d^2s}{dt^2} \curvearrowright \overset{t}{\underset{0}{\int}} \dfrac{F}{m} \cdot dt^2 = \overset{s_n}{\underset{s_1}{\int}} d^2s\\
\notag \curvearrowright \\
s &= f(t)
\end{align}
\textcolor{red}{Was sagt mir das hier? ZUsammenhang nicht klar.}
ABER man benötigt noch die wirkenden Kräfte. Diese bekommt man über die Berechnung von $\dfrac{dE}{dx}=F$, dafür muss die Schrödingergleichung gelöst werden (ab initio Moleküldynamik) oder man berechnet die Energie mittels Molekülmechanik. Der Zusammenhang zwischen Geschwindigkeit / kinetischer Energie und Temperatur ist durch folgende Gleichung gegegben:
\begin{align}
\dfrac{1}{2} m v^2 &= \dfrac{3}{2} k_b T\\
\notag k_b &- \text{Boltzmann-Konstante}
\end{align}
Die so berechnete kinetische Energie, sowie deren Richtung werden stochastisch auf alle Teilchen verteilt $\curvearrowright$ Atome schwingen um ihre Gleichgewichtspotentiale. Die resultierende Dynamik muss im fs-Bereich integriert werden, um chemisch interessante Sachverhalte zu beobachten. In der Moleküldynamik werden die chemischen Eigenschaften zum zeitlichen Mittelwert (man weiß nie wann eine Eigenschaft für ein Molekül gilt).
\paragraph{Monte-Carlo-Ansatz}
Man ändert die Atompositionen stochastisch und berechnet dann aus den erhaltenen Konformationen eine Boltzmann-Verteilung um die gemittelten Werte zu erhalten.
\paragraph{Konformationssuche mittels Moleküldynamik} 
Erhöhung der Temperatur eines Systems und dann alle $\Delta$t eine Struktur auswählen und minimieren, so findet man viele Minimalenergiekonformationen.


\section{Comparative Modeling}
\subsection{Multiple Sequence Alignment}
\subsection{Structural Alignment vs. Sequence}

\subsection{Protein fold recognition}



\end{document}