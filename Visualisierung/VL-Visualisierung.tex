\documentclass[12pt,a4paper,oneside,normalheadings,abstracton,liststotoc,bibtotoc,titlepage,pdftex]{scrartcl}
%\usepackage {ngerman}
\usepackage [ngerman] {babel}
\usepackage[utf8]{inputenc}
\usepackage{float}
\usepackage{graphicx}
\usepackage{longtable}
\usepackage{tabularx}
\usepackage{setspace}
\usepackage{upgreek}
\usepackage{color}
\usepackage{subfig}
\usepackage{gensymb}
\usepackage{amsmath,amsfonts,amssymb}
\usepackage{hyperref}
% math. Symbole und Umgebungen
\onehalfspacing

\begin{document}
\pagenumbering{Roman}

%\extratitle{~}
\titlehead{Universität Leipzig\\Fakultät für Fakultät für Mathematik und Informatik}
\subject{Vorlesungsmitschrift SoSe 11}
\title{Visualisierung}
\author{gehalten von Prof. Dr. Gerik Scheuermann und Dr. Alexander Wiebel (wiebel@informatik.uni-leipzig.de)\\
Mitschrift von Tobias Mede und Christoph Kämpf}
\date{\today}

\maketitle[1]

\begin{abstract}

\end{abstract}

\tableofcontents
\pagebreak

\pagenumbering{arabic}

\chapter{Visualisierung in Naturwissenschaft und Technik}

\section{Einführung}
Scientific Visualization ist das womit wir uns beschäftigen.
Visualisierungspipeline:


\section{Datenrepräsentation}
\textbf{Prüfungsrelevant:} Aufbau der Datensätze\\


Drei Teile:
-Definitionsmenge = Teilmenge des Beobachtungsraums B$^b$
-Nachbarschaftsrelation
-Funktion der Definitionsmenge

!!Metrik
\subsection{Voronoidiagramm}
Maximal 2n-5 Ecken und 3n-6 Kanten. Delaunay Triangulierung.

\section{Vektoranalysis}


Integrationskurven können sich nicht schneiden.
\textbf{Prüfungsrelevant:} Eulerverfahren\\
\textbf{Prüfungsrelevant:} Schrittweitenkontrolle bei \textit{Step-Doubling}\\



\section{Tensoren}
Tensoren 0. Ordnung sind Skalare.\\
Tensoren 1. Ordnung sind Vektoren.\\
Tensoren 2. Ordnung sind Matrizen.\\
Die Ableitung eines Tensors x. Ordnung ergibt einen Tensor der Ordnung x+1. 

\textbf{Prüfungsrelevant (nur wenn noch Zeit ist lernen):} Strömungsdynamik\\

\section{Analysemethoden}
Barizentrische Koordinaten eines 1-, 2- oder 3-dimensionalen Simplex. Ein Simplex ist eine Menge von Punkten \dots . 



\section{Vektorfeldtopologie}
$\upalpha$-Becken umfasst alle Punkte deren Stromlinien aus derselben Quelle (Punkt, Linie, etc.) stammen. $\upomega$-Becken umfasst alle Punkte deren Stromlinien die selbe Senke (Punkt, Linie, etc.) haben.

\subsection{Konturbaum}

\end{document}