\documentclass[12pt,a4paper,oneside,normalheadings,abstracton,liststotoc,bibtotoc,titlepage,pdftex]{scrbook}
%\usepackage {ngerman}
\usepackage [ngerman] {babel}
\usepackage[utf8]{inputenc}
\usepackage{float}
\usepackage{graphicx}
\usepackage{longtable}
\usepackage{tabularx}
\usepackage{setspace}
\usepackage{upgreek}
\usepackage{color}
\usepackage{subfig}
\usepackage{gensymb}
\usepackage{amsmath,amsfonts,amssymb}
\usepackage{hyperref}
% math. Symbole und Umgebungen
\onehalfspacing

\begin{document}
\pagenumbering{Roman}

%\extratitle{~}
\titlehead{Universität Leipzig\\Fakultät für Fakultät für Mathematik und Informatik}
\subject{Vorlesungsmitschrift WiSe 2010}
\title{Visualisierung}
\author{gehalten von Prof. Dr. Gerik Scheuermann und Dr. Alexander Wiebel\\
Mitschrift von Tobias Mede und Christoph Kämpf}
\date{\today}

\maketitle[1]


\tableofcontents
\pagebreak

\pagenumbering{arabic}

\chapter{Visualisierung in Naturwissenschaft und Technik}

\section{Einführung}
Scientific Visualization ist das womit wir uns beschäftigen.
Visualisierungspipeline:


\section{Datenrepräsentation}
\textbf{Prüfungsrelevant:} Aufbau der Datensätze\\
Drei Teile:
-Definitionsmenge = Teilmenge des Beobachtungsraums B$^b$
-Nachbarschaftsrelation
-Funktion der Definitionsmenge

!!Metrik
\subsection{Voronoidiagramm}
Maximal 2n-5 Ecken und 3n-6 Kanten. Delaunay Triangulierung.

\section{Vektoranalysis}

\section{Direkte Visualisierung}
\subsection{Visualisierung von Vektordaten}
\subsubsection{Experimentelle Strömungsvisualisierung}
Eine Möglichkeit der experimentellen Strömungsvisualisierung ist das Einbringen von sichtbarem Fremdmaterial in die Strömung. Dabei unterscheidet man:
\begin{description}
\item[\textit{Time lines}:] Partikel werden zur gleichen Zeit (z.B. entlang einer Linie) in die Strömung eingebracht
\item[\textit{Streak lines}:] Partikel werden an einer fixen Position (z.B. mittels Düsen am umströmten Objekt) in die Strömung eingebracht
\item[\textit{Path lines}:] Partikel werden statistisch verteilt in Strömung eingebracht und ihr Weg für eine kurze Zeit (z.B. lange Belichtungszeit von leuchtenden Partikeln) nachverfolgt
\end{description}

\subsubsection{Strömungsvisualisierung gemessener Daten}
Um einen schnellen Eindruck vom gemessen Vektorfeld zu bekommen bietet sich die \textit{Hedgehog}-Technik an.


\end{document}